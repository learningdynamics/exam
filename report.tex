\documentclass[a4paper, 12pt]{article}
\usepackage{natbib,alifexi}

\usepackage[T1]{fontenc}
\usepackage[utf8]{inputenc}

\usepackage{times}

\usepackage{mathtools}
\usepackage{booktabs}
\usepackage{graphicx}

\usepackage{algpseudocode}
\usepackage{algorithm}

\title{Reproducing the result of the paper: ``Local Coordination in Online Distributed Constraint Optimization Problems'' \citep{Brys2012} \\ \large Learning dynamics (INFO-F-409)}
\author{Leon Babin$^1$, Yanfang Guo$^2$, Anthony Piron$^1$ \and Wenjia Wang$^2$ \\
\mbox{}\\
$^1$Master in Bio-informatics and Modelling, Université Libre de Bruxelles \\
$^2$Vrije Universiteit Brussel}


\begin{document}
\maketitle

\begin{abstract}
  
\end{abstract}

\section{Introduction}


\section{LJAL}

\begin{algorithm}[tph]
  \caption{The LJAL algorithm. Update Qs. Return reward.}
  \label{alg:ljal}
  \begin{algorithmic}[1]
    \Function{LJAL\_One\_Step}{agents, graph}
    \ForAll{agent in agents}
    \State $evs\gets EVs(graph.nodes[agent], agent)$
    \State $actions[agent]\gets BoltzmannAction(evs)$
    \EndFor

    \State $R\gets Reward(actions)$

    \ForAll{node in graph.nodes}
    \State $sel.actions\gets actions[node.successors]$
    \State $node.Q[sel.actions] \mathrel{+}\gets$
    \State \hspace{1cm}$\alpha * (R - node.Q[sel.actions])$
    \State $node.N[sel.actions]\mathrel{+}\gets 1$
    \EndFor
    \State \Return $R$
    \EndFunction
    \State
    \ForAll{t in MaxTime}
    \State \Call{LJAL\_One\_Step}{agents, graph}
    \EndFor
  \end{algorithmic}
\end{algorithm}

The LJAL algorithm is provided at \ref{alg:ljal}. A graph is a list of nodes. Each node contains a list of successors, a $Q$ matrice (size $\bigtimes_{s \in successors} a_s$ where $a_s$ is the number of actions for the successor $s$) and $N$ a matrix (same size) for counting the number of time an action is chosen. 

\begin{figure*}[tp]
  \centering
  \includegraphics[scale=0.8]{part1_plot.png}
  \caption{Comparison of independent learners, joint action learns and local joint action learners on a distributed bandit problem.}
  \label{fig:part1}
\end{figure*}

\begin{table*}[tp]
  \centering
  \include{part1_table}
  \caption{Distributed bandit problem: Comparison of speed and solution quality.}
  \label{tab:part1}
\end{table*}


\section{DCOP}

\begin{figure*}[tp]
  \centering
  \includegraphics[scale=0.8]{part2_plot.png}
  \caption{Comparison of independent learners, joint action learns and local joint action learners on a distributed constraint optimization problem.}
  \label{fig:part2}
\end{figure*}

\begin{table*}[tp]
  \centering
  \include{part2_table}
  \caption{distributed constraint optimization problem: comparison of speed and solution quality.}
  \label{tab:part2}
\end{table*}

\section{Learning coordination graphs}

\begin{algorithm}[tph]
  \caption{Coordination graph learning algorithm.}
  \label{alg:LCG}
  \begin{algorithmic}
    \Function{LCG\_One\_Step}{agents}
    
    \EndFunction
  \end{algorithmic}
\end{algorithm}





\section{Acknowledgements}

\footnotesize
\bibliographystyle{apalike}
\bibliography{bib}


\end{document}
